% This is a simple template for a LaTeX document using the "article" class.
% See "book", "report", "letter" for other types of document.

\documentclass[10pt,titlepage]{article} % use larger type; default would be 10pt
\raggedright

\usepackage{tgadventor}
\renewcommand*\familydefault{\sfdefault} %% Only if the base font of the document is to be sans serif
\usepackage[T1]{fontenc}

%%% PAGE DIMENSIONS
\usepackage{geometry} % to change the page dimensions
\geometry{
	letterpaper,
	right = 1.0in,
	left = 1.0in,
	top = 1.5in,
	bottom = 0.59in,
    head=0.7in,
    includefoot = true,
}

%%% PACKAGES
\usepackage{array} % for better arrays (eg matrices) in maths
\usepackage{paralist} % very flexible & customisable lists (eg. enumerate/itemize, etc.)
\usepackage{verbatim} % adds environment for commenting out blocks of text & for better verbatim
\usepackage{subfig} % make it possible to include more than one captioned figure/table in a single float
\usepackage{multicol}
\usepackage{graphicx} % support the \includegraphics command and options
\usepackage[parfill]{parskip} % Begin paragraphs with an empty line rather than an indent


%%% Acronym setup
\usepackage{acro}
\acsetup{
    make-links,
    format/short    = \textsc,
    pages / display = first, % first|all|none
    list / template = descriptionTable, % description|table|longtable|lof|toc
    list / display  = all, % all|used,
    list / name     = Definitions,
    list / sort     = true,
}

\NewAcroTemplate[list]{descriptionTable}{%
    \acronymsmapT{%
        \AcroAddRow{%
            \bfseries \acrowrite{short}%
            &
            \acrowrite{long}%
            &
            \acrowrite{extra}%
            &
            \acrowrite{tag}%
            &
            \acropages
                {\acrotranslate{page}\nobreakspace}%
                {\acrotranslate{pages}\nobreakspace}%
            \tabularnewline
        }%
    }%
    %    \acroheading
    %    \acropreamble
    \par\noindent
    %\setlength\LTright{0pt}%
    \rowcolors{2}{myMedBlue}{myLightBlue}
    \begin{longtable}{@{}l L{1.75in} L{2.3in} L{1in} r@{}}
        \rowcolor{myDarkBlue}
        {\color{white} \textbf{Abbr.}} &
        {\color{white} \textbf{Term}} &
        {\color{white} \textbf{Description}} &
        {\color{white} \textbf{Category}} &
        {\color{white} \textbf{Pg.}} \\
        \endfirsthead
        %
        \multicolumn{5}{c}%
        {{\bfseries Definitions continued from previous page}} \\
        \rowcolor{myDarkBlue}
        {\color{white} \textbf{Abbr.}} &
        {\color{white} \textbf{Term}} &
        {\color{white} \textbf{Description}} &
        {\color{white} \textbf{Category}} &
        {\color{white} \textbf{Pg.}} \\
        \endhead
        %
        \AcronymTable
    \end{longtable}
}

%%% HEADERS & FOOTERS
\usepackage{fancyhdr} % This should be set AFTER setting up the page geometry
\usepackage{lastpage}
\pagestyle{fancy} % options: empty , plain , fancy
\renewcommand{\headrulewidth}{0pt} % customise the layout...
\renewcommand{\footrulewidth}{0.5pt} % customise the layout...


%%% SECTION TITLE APPEARANCE
\usepackage{titlesec}
\titlespacing{\section}{0pt}{*3}{*2}
\titleformat{\section}[hang]
    {\bfseries{\Large}}
    {\Large\thesection}{10pt}{\Large\sc}


%%% ToC (table of contents) APPEARANCE
\renewcommand\contentsname{Table of Contents}
\setcounter{tocdepth}{2}

\usepackage[nottoc,notlof,notlot,numbib]{tocbibind} % Put the bibliography in the ToC
\usepackage[titles,subfigure]{tocloft} % Alter the style of the Table of Contents

%% ToC settings for Section
\renewcommand{\cftsecfont}{\scshape}
\renewcommand{\cftsecpagefont}{\upshape} % No bold!
\renewcommand{\cftsecleader}{\cftdotfill{\cftsecdotsep}}
\renewcommand\cftsecdotsep{\cftdot}

%% ToC settings for Subsection
\renewcommand{\cftsubsecfont}{\scshape}
\renewcommand{\cftsubsecpagefont}{\upshape} % No bold!
\renewcommand{\cftsubsecleader}{\cftdotfill{\cftsecdotsep}}
\renewcommand\cftsubsecdotsep{\cftdot}

%% LoF settings for Figure
\renewcommand{\cftfigfont}{\scshape Fig. }
\renewcommand{\cftfigpagefont}{\upshape} % No bold!
\renewcommand{\cftfigleader}{\cftdotfill{\cftsecdotsep}}
\renewcommand\cftfigdotsep{\cftdot}
\setlength{\cftfigindent}{0pt}

%% LoT settings for Table
\renewcommand{\cfttabfont}{\scshape Tab. }
\renewcommand{\cfttabpagefont}{\upshape} % No bold!
\renewcommand{\cfttableader}{\cftdotfill{\cftsecdotsep}}
\renewcommand\cfttabdotsep{\cftdot}
\setlength{\cfttabindent}{0pt}


%%% Support for tables and colored rows
\usepackage[hyperref,table,xcdraw,x11names]{xcolor}
\usepackage{longtable}
\usepackage{booktabs}
\usepackage{multirow}

% Define ragged right/left and centering for fixed width table columns
\newcolumntype{L}[1]{>{\raggedright\let\newline\\\arraybackslash\hspace{0pt}}p{#1}}
\newcolumntype{C}[1]{>{\centering\let\newline\\\arraybackslash\hspace{0pt}}m{#1}}
\newcolumntype{R}[1]{>{\raggedleft\let\newline\\\arraybackslash\hspace{0pt}}p{#1}}

% Spacing between table rows
\renewcommand{\arraystretch}{1.5}

% Table colors
\definecolor{myDarkBlue}{HTML}{4F81BD}
\definecolor{myMedBlue}{HTML}{DEEAF6}
\definecolor{myLightBlue}{HTML}{F3F6FF}


\usepackage{hyperref}
\hypersetup{
    colorlinks=true,
    linkcolor=blue,
    filecolor=magenta,
    urlcolor=cyan,
    pdftitle={Overleaf Example},
    pdfpagemode=Bookmarks,
    linktocpage=false,
    hyperindex=true,
    pdftitle=\titleStr,
    pdfauthor=\departmentOwner,
}

\urlstyle{same}
%%% END Article customizations